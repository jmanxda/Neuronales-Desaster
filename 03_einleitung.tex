% !TEX root = 00_arbeit.tex

%---------------------------------------------------------------------------------
%% Einleitung

\section{Einleitung}



Die heutige Gesellschaft ist geprägt durch eine zunehmende Digitalisierung, durch immer schneller voranschreitenden technologischen Wandel und einer globaleren Vernetzung. Wir als Menschen sind in der Lage Kenngrößen der Natur aufzunehmen, die nicht länger auf unsere menschlichen Sinnesorgane begrenzt sind. Wir sammeln kontinuierlich riesige Datenmengen auf der ganzen Welt und darüber hinaus. Der hierdurch verursachte Daten- und Informationsstrom überschreitet bei Weitem unsere menschliche Aufnahmefähigkeit.

Um den Überblick zu bewahren, sind wir zunehmend auf computerbasierte Datenverarbeitung angewiesen, die die Komplexität auf ein für uns verarbeitbares Maß reduziert. Hierbei liegt die größte Stärke des Computers in der schnellen, effizienten und fehlerreduzierten Berechnung von großen Zahlen- und Datenmengen. Ein Taschenrechner berechnet eine komplexe mathematische Funktion im Bruchteil einer Sekunde, während unser Gehirn für die gleiche Aufgabe wesentlich länger braucht. Um diese Leistung zu erbringen, braucht ein Computer jedoch klare Handlungsanweisungen, die auch als Algorithmen bezeichnet werden. Diese müssen vom Menschen vorformuliert und in ein mathematisches Modell überführt werden. Das setzt wiederum voraus, dass der Anwendungskontext klar und präzise abgesteckt ist und alle Einflussfaktoren und Regeln zuvor bekannt sind.

Die größten Herausforderungen unserer Zeit liegen jedoch im Unbekannten, in den Tiefen der neu gewonnen Datenquellen. Indem die Daten in einen Kontext gebracht werden, entstehen Informationen. Die Daten haben nun an Bedeutung gewonnen und verfügen über eine gewisse Semantik. Die Summe aller Informationen zu einem bestimmten Sachverhalt stellt das Wissen dar. Es bildet das Fundament für unsere Fähigkeit Entscheidungen zu treffen und Probleme zu lösen. Somit beeinflusst es maßgeblich unser Denken und Handeln. Im Laufe der Evolution hat sich unser Gehirn zu einem hochkomplexen System entwickelt, dass Sinneseindrücke über verschiedenste Kanäle zusammenträgt, zu Informationen weiterverarbeitet und als Wissen abspeichert, um es in Zukunft noch schneller abrufen zu können. Wir nennen diesen Prozess gemein hin Lernen.

Welche Möglichkeiten gibt es, um Systeme zu erschaffen, die nicht nur schneller, effizienter und rastloser arbeiten als wir, sondern zusätzlich über die Fähigkeit verfügen Zusammenhänge zu erkennen und zu lernen? Die aktuelle Forschung im Bereich der so genannten künstlichen Intelligenz befasst sich unter anderem mit dieser Fragestellung. Die Idee, ein künstliches System in Analogie zu unserem menschlichen Gehirn zu erschaffen, um so den Computer zum Lernen zu bringen, wird derzeit intensiv erforscht. Künstliche neuronale Netze greifen genau diese Idee auf und gelten als eine der vielversprechendsten Ansätze, um die Komplexität unserer heutigen Welt auf ein Maß zu reduzieren, das für uns handhabbar wird.

Ein Beispiel für eine solche Problemstellung stellt die Vorhersage des Strompreises dar. Der Verlauf des Strompreises ist gekennzeichnet durch hohe Schwankungen, die durch vielfältige Einflüsse hervorgerufen werden. 


Einige dieser Einflussfaktoren können wir zwar erahnen aber es bedarf viel Erfahrung, um den gegenseitigen Einfluss der Faktoren abschätzen zu können. Durch diese Erfahrung ist es Spezialisten möglich eine Prognose für die Strompreisentwicklung zu geben. Die Anzahl an Menschen die über dieses hoch spezielle Wissen und Erfahrung verfügen ist begrenzt. Dem gegenüber steht eine große Anzahl an Marktteilnehmern die an präzisen Preisprognosen interessiert sind. Dies sind sowohl Produzenten wie auch Konsumenten die die zukünftige Preisgestaltung nutzen, um ihre Produktionspläne anzupassen und passende Strategien zu entwickeln wie sie ihren Nutzen maximieren.

Aus diesem Grund beschäftigt sich diese Arbeit mit der Anwendung künstlicher neuronaler Netze zur Prognose von Strompreisen. Hierzu wird zunächst ein Einblick in den Aufbau des menschlichen Gehirns gegeben. Nachfolgend wird die Funktion des biologischen und künstlichen Neurons gegenübergestellt. Auf dieser Grundlage werden aktuelle Modelle künstlicher neuronaler Netze vorgestellt und hinsichtlich ihrer Anwendungsgebiete eingeordnet. 

Als nächstes werden die Eigenheiten des Strommarktes und die Besonderheiten des Strompreises vorgestellt. Daraufhin erfolgt eine Literaturrecherche im Hinblick auf Anwendung künstlicher neuronaler Netze zur Vorhersage des Strompreises. Hierbei werden auch die Maße welche in der Literatur zur Evaluierung der Vorhersagegenauigkeit eingesetzt werden näher betrachtet. 

Unter Berücksichtigung der Literaturrecherche werden im nächsten Schritt ausgewählte Algorithmen detailliert vorgestellt und diskutiert. Schließlich werden die Algorithmen in Stata implementiert und Prognosen anhand von empirischen Strommarktdaten erstellt. Abschließend werden die Vorhersageergebnisse analysiert.


\newpage

% Der Theorieteil wird informativ gehalten wobei ausgewählte Algorithmen im Nachgang mathematisch vorgestellt werden.