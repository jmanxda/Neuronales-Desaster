        %---------------------------------------------------------------
        %% SLP
        %-----------------------------------------
        %% linkes Bild
        \begin{tikzpicture}[>=stealth', node distance=\layersep cm, shorten >=1pt]
        \def\layersep{2}            % vertikal distance between the layers
        \def\neuronsep{1.5}         % Horizontal distance between neurons
        \def\dlsize{1.5}            % distance between node and layer lable
        \def\inout{\layersep*.65}   % Size of in- and output-arrow
        \def\siz{.8}                % neuronsize
        \def\y{3}                   % Start of the most upper layer
        \def\ni{5}                  % Amount of input neurons
        \def\no{2}                  % Amount of output neurons
        \tikzstyle{neuron}=[circle,draw=black,minimum size=\siz cm,inner sep=2pt]
        \tikzstyle{annot} = [text width=5em, text centered]
        \tikzset{fontscale/.style = {font={\fontsize{#1pt}{#1pt}\selectfont}}}
        \newcommand{\neurono}[2][]{
            \node[neuron,circle split,inner sep=2pt] (#1) at (#2)
                    {\includegraphics[width=0.225cm]{Bilder/Sigma.png} \nodepart{lower} \includegraphics[width=0.225cm]{Bilder/sigma.png}};
        }

        % Draw the left input layer nodes
            \foreach \name / \xn in {1,...,\ni}{
            % This is the same as writing \foreach \name / \y in {1/1,2/2,3/3,4/4}
                \node[neuron,fontscale=15] (Il-\name) at (\xn*\neuronsep-\neuronsep,\y) {$i_{\xn}$};
                \node[above of=Il-\name, node distance=\inout cm] (Inl-\name) {};
                \draw [->,arrows={-Stealth[length=7pt]},densely dotted] (Inl-\name) edge (Il-\name);
            }
        % Draw the output layer node
            \foreach \name / \xn in {1,...,\no}{
                \node[neuron,fontscale=15] (Ol-\xn) at ({(\ni-1)*\neuronsep/2-\neuronsep/2*(\no-1)+(\xn-1)*\neuronsep},\y-\layersep) {$\Omega_{\xn}$};
                \node[node distance=\inout cm, below of=Ol-\xn] (Onl) {};
                \draw [->,arrows={-Stealth[length=7pt]},densely dotted] (Ol-\xn) edge (Onl);
        % Connect every node in the input layer with the output layer
            \foreach \source in {1,...,\ni}
                \draw [->,arrows={-Stealth[length=7pt]}] (Il-\source) edge (Ol-\xn);
                }
        % Annotate the layers
                \node[annot,right of=Il-\ni, node distance=\dlsize cm] (il) {\textbf{Eingabe- schicht}};
                \node[annot,below of=il] {\textbf{Ausgabe- schicht}};
        %-----------------------------------------
        %% rechtes Bild
            \tikzset{
                ident/.pic={
                \draw[semithick] (-\siz/4,-\siz/4) -- (\siz/4,\siz/4);
            }}
        % Draw the right input layer nodes
                \node[xshift=\dlsize cm] (Ir) at (il) {};
            \foreach \name / \xn in {1,...,\ni}{
        % This is the same as writing \foreach \name / \y in {1/1,2/2,3/3,4/4}
                \node[neuron] (Ir-\name) at ($(Ir)+(\xn*\neuronsep-\neuronsep,0)$) {};
                \node[above of=Ir-\name, node distance=\inout cm] (Inr-\name) {};
                \pic at (Ir-\name) {ident};
                \draw [->,arrows={-Stealth[length=7pt]},densely dotted] (Inr-\name) edge (Ir-\name);
            } 
        % Draw the output layer node
            \foreach \name / \xn in {1,...,\no}{
                \neurono[Or-\xn]{$(Ir)+({(\ni-1)*\neuronsep/2-\neuronsep/2*(\no-1)+(\xn-1)*\neuronsep},-\layersep)$}

                \node[node distance=\inout cm, below of=Or-\xn] (Onr) {};
                
                \draw [->,arrows={-Stealth[length=7pt]},densely dotted] (Or-\xn) edge (Onr);
                
        % Connect every node in the input layer with the output layer
            \foreach \source in {1,...,\ni}
                \draw [->,arrows={-Stealth[length=7pt]},every node/.style={fill=white,inner sep=1pt,fontscale=7}] 
                (Ir-\source) edge  (Or-\xn) 
                node at ($(Ir-\source)!.3+.18*(\xn-1)!(Or-\xn)$) {$w_{\source,\xn}$};
            }
\end{tikzpicture}