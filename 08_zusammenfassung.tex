% !TEX root = 00_arbeit.tex

%---------------------------------------------------------------------------------
%% Zusammenfassung

\section{Zusammenfassung und Ausblick} 

\begin{sloppypar}
Ziel dieser Arbeit war es, die Anwendbarkeit künstlicher neuronaler Netze auf die Prognose von Strompreisen zu untersuchen. Für diese Aufgabe wurde zunächst der Aufbau und die Funktionsweise des menschlichen Gehirns als Vorbild für neuronale Netze betrachtet. Im Anschluss wurden biologische und künstliche Neurone einander gegenübergestellt. Als nächstes wurden Charakterisierungsmöglichkeiten künstlicher neuronaler Netze aufgezeigt, aktuelle Modelle vorgestellt und hinsichtlich ihrer Einsatzgebiete eingeordnet. Die Eigenheiten des Strommarktes und die Besonderheiten des Strompreises wurden nachfolgend erläutert. Anschließend wurde ein Literaturüberblick über die Anwendung künstlicher neuronaler Netze zur Vorhersage des Strompreises im Zeitraum zwischen 2016 und 2018 gegeben. Die in der Literatur angewandten Maße zur Evaluierung der Vorhersagegenauigkeit wurden daraufhin diskutiert. Dabei wurden die am häufigsten angewandten Algorithmen zur Strompreisvorhersage hergeleitet und gegenübergestellt. Abschließend wurde der zur Analyse der Algorithmen genutzte Datensatz vorgestellt und die Prognosegenauigkeit der Algorithmen analysiert.\par\medskip 

Die Charakterisierungsmöglichkeiten künstlicher neuronaler Netze sind vielfältig. In dieser Arbeit wurden verschiedene Netzwerke anhand der eingesetzten Neuronen, ihrer Verbindungsarchitektur und des überwiegend eingesetzten Lernalgorithmus vorgestellt. Es stellte sich heraus, dass die Einsatzgebiete ebenso vielfältig ausfallen können, jedoch gibt es auch Netzwerke wie das Hopfieldnetz, die sehr spezielle Anwendungen besitzen. Allgemein wurden die Einsatzgebiete eingeteilt in Klassifizierung, kombinatorische Optimierung sowie assoziatives Gedächtnis, wobei die Einteilung nicht als starr und klar umgrenzt angesehen werden kann. Aufgrund der Vielseitigkeit künstlicher neuronaler Netze können sie nicht direkt, sondern nur innerhalb der gleichen Anwendung miteinander verglichen werden. %Vielfältig sind ebenfalls die Einsatzgebiete aber es gibt auch Netzwerke, wie das Hopfieldnetz, die sehr spezielle Anwendungen besitzen.
\par\medskip 

Der Strompreis ist gekennzeichnet durch Volatilität, Saisonalität, Preis-Spitzen, Sprünge und Nichtlinearitäten und kann daher nur schwer mit ökonometrischen Modellen vorhergesagt werden. Es wurde gezeigt, dass es in der Literatur unterschiedliche Herangehensweisen und Modelle gibt, die den Versuch unternehmen den Strompreis zu prognostizieren. Die vielversprechendsten Ansätze liefern hierbei hybride Modelle, da sie unterschiedliche Muster in den Daten erfassen können und somit zu einer geringeren Vorhersageabweichung führen. Da der Fokus dieser Arbeit auf den künstlichen neuronalen Netzen lag, wurde bei der Literaturrecherche auch beim Einsatz hybrider Modelle nur das angewandte neuronale Netz betrachtet. Die Literaturrecherche im genannten Zeitraum ergab, dass das MLP das am häufigsten zur Strompreisvorhersage eingesetzte Netzwerk ist. Darüber hinaus wurde herausgefunden, dass als Lernalgorithmen das Backpropagation-Verfahren und der Levenberg-Marquardt Algorithmus am häufigsten zur Anwendung kommen. Schließlich lassen die Ergebnisse der Veröffentlichungen durch den Einsatz unterschiedlicher Datensätze keinen Schluss über die beste Netzwerkkonstellation zur generellen Prognose von Strompreisen zu. Ein gemeinsamer Datensatz bei zukünftigen Arbeiten würde eine Vergleichbarkeit der unterschiedlichen Modelle ermöglichen.\par\medskip 

Um herauszufinden, welche Netzwerkkonstellationen zu einer besseren Prognose führen und ob der Lernalgorithmus ebenfalls einen Einfluss auf die Prognosefähigkeit eines Netzwerkes ausübt, wurde zunächst das Backpropagation-Verfahren und der Levenberg-Marquardt Algorithmus hergeleitet und gegenübergestellt. Die Gegenüberstellung der Lernalgorithmen ergab, dass das BP-Verfahren im Vergleich zum LM-Algorithmus simpler aufgebaut ist. Dafür arbeitet der LM-Algorithmus nach der Initialisierung autonom, im Gegensatz zum BP-Verfahren, bei dem die optimalen Parameter iterativ bestimmt werden müssen. Da der LM-Algorithmus Matritzen enthält, welche die Gesamtheit der Gewichte sowie Trainingsbeispiele beinhalten, benötigt dieser Algorithmus bei großen Datensätzen einen hohen Speicherbedarf. Der zu untersuchende Datensatz beinhaltet im Trainingsset 46.180 Datenpunkte und wurde als groß angesehen. Um möglichen Berechnungsproblemen vorzubeugen und den Speicherbedarf zu reduzieren, wurde der Algorithmus bei der Implementierung dahingehend abgewandelt, dass für jede Gewichtsschicht eine eigene Matrix erstellt wurde.\par\medskip

Es folgt die Untersuchung der Prognosefähigkeit des MLP-Netzes. Zusätzlich zu den Lernalgorithmen wurden als Netzwerkkonstellationen die logistische Funktion und der Tangens-Hyperbolicus als Aktivierungsfunktion betrachtet und die An- bzw. Abwesenheit von Bias-Neuronen berücksichtigt. Unter diesem Gesichtspunkt kann die in der Literatur zu findende Aussage nicht bestätigt werden, wonach der LM-Algorithmus schneller lernt. Des Weiteren führte der Einsatz von Bias-Neuronen zu höheren Optimierungszeiten und zeigte keinen positiven Einfluss auf die Vorhersagegenauigkeit. Die Ergebnisse der Analysen konnten keine generelle Überlegenheit eines Algorithmus zeigen.\par\medskip  %Anschließend wurde die Prognosefähigkeit des MLP-Netzes untersucht.

\newpage

Abschließend lassen die Ergebnisse für die Anwendbarkeit von Mehrschichtperceptrons zur Strompreisvorhersage folgenden Schluss zu: Bei der Untersuchung eines gleichbleibenden Datensatzes, das heißt es werden weder Variablen hinzugefügt noch entfernt, ist das Backpropagation-Verfahren durch die niedrigere Berechnungszeit zu bevorzugen. Werden aber unterschiedliche Datensätze untersucht, ist der Levenberg-Marquardt Algorithmus durch die geringere Optimierungszeit vorteilhafter. Bei beiden Algorithmen ist durch die deutlich erhöhte Optimierungszeit von einem Einsatz von Bias-Neuronen abzuraten.\par\medskip 

Da nicht eindeutig ausgeschlossen werden kann, dass die Netzwerke auf die Testdaten zugeschnitten worden sind, wird für zukünftige Arbeiten ein weiteres Validierungsset empfohlen. Weiterhin kann die Prognose verbessert werden, indem Variablen dem Datensatz hinzugefügt oder entfernt werden. Eine weitere Möglichkeit die Prognosegenauigkeit zu verbessern, ist mehrere Vorhersagemodelle sowie weiterer CI-Methoden zu hybriden Modellen \hbox{zusammenzuschließen}.\par\medskip 
\end{sloppypar}
%----Ausblick

%\todo{gemeinsamer Datensatz um Ergebnisse untereinander vergleichen zu können}

%\farbig{Da aber die Ergebnisse ebenfalls zeigen, dass die Berechnungszeit eines einzelnen Netzwerkes beim Vorhandensein von Bias-Neuronen auch geringer ausfallen kann als ohne sollte der Einfluss der Bias-Neuronen mit einer größeren Stichprobe in einer Nachfolgearbeit untersucht werden.} 