%-------------------------------------------
%% Zusätzliche Pakete werden geladen
%\documentclass[a4paper,12pt,titlepage]{article}  
\documentclass[a4paper, 12pt, titlepage]{scrartcl} % Schrift- und Papiergröße werden eingestellt  
\usepackage[top=3cm, bottom=3cm, left=2.5cm, right=2.5cm]{geometry} % Korrekturrand wird eingestellt
\usepackage[ngerman]{babel}             % Dokumentsprache wird auf deutsch gesetzt
\usepackage[onehalfspacing]{setspace}   % Zeilenabstand wird auf 1,5 gesetzt
\usepackage{color}                      % Ermöglicht die Änderung der Schriftfarbe
\usepackage{graphicx}                   % Ermöglicht das Einbinden von Bildern
\usepackage{fancyhdr}                   % Ermöglicht die freie Bearbeitung der Fuß und Kopfzeilen
\usepackage{tocloft}                    % Ermöglicht das anpassen des Inhaltsverzeichnisses
\usepackage{wrapfig}                    % Ermöglicht textumschlossene Abbildungen
\usepackage{array}                      % Ermöglicht das Anpassen von Tabellen

%-------------------------------------------
%% Fonteinstellungen abhängig vom Compiler
%% Quellen: https://en.wikibooks.org/wiki/LaTeX/Fonts
%%          https://www.sharelatex.com/learn/XeLaTeX
%%          https://tex.stackexchange.com/questions/12565/load-fonts-that-are-in-a-fonts-directory
\usepackage{ifxetex}                    % Ermöglicht das Prüfen ob XeLaTeX benutz wird
    \ifxetex
    \usepackage{fontspec}
    \defaultfontfeatures{Ligatures=TeX} % To support LaTeX quoting style
    \setmainfont[%                      % Times New Roman (aus Windows) wird als Standardschrift gesetzt
        Path = Fonts/,%                 % Dateien Pfad wird auf Fonts/ gesetzt
        BoldFont=timesbd.ttf,%
        ItalicFont=timesi.ttf,%
        BoldItalicFont=timesbi.ttf,%
        ]{times.ttf}
\else
    \usepackage[T1]{fontenc}            % Liefert die meisten Text-Zeichen für westeuropäische Sprachen
    \usepackage[utf8]{inputenc}         % Ermöglicht die direkte Eingabe der Umlaute
    \usepackage{mathptmx}               % Times wird als Standardschrift gesetzt
\fi

%-------------------------------------------
%% Pakete für das Erstellen von Refferenzen
\usepackage{varioref}
\usepackage{hyperref} % Erweitert die Referenzmöglichkeiten
\usepackage{cleveref}

%-------------------------------------------
%% Pakete für das Erstellen des Literaturverzeichnisses
\usepackage[babel,german=quotes]{csquotes}  % Verwendet die deutsche Art der Anführungszeichen
\usepackage[backend=biber, style=authoryear, citestyle=authoryear, maxitems=1]{biblatex}

\DefineBibliographyStrings{german}{%    % Verändert beim zitieren "u.a." zu "et al." und "und" zu "and"
  andothers = {et al.},
  and       = {and}
}
\addbibresource{10_literatur.bib}       % Fügt die Literaturverzeichnisdatei ein
\setlength{\bibitemsep}{10pt}           % Stellt den Zeilenabstand zwischen den Literaturverzeichnis Einträgen ein
%-------------
%% Erstellt ein Makro welches prüft ob die eingegebene Variable leer ist
\def\ifempty#1{% 
\def\temp{#1}%
\ifx\temp\empty%
}
%-------------
%% Erstellt mit \fcite ein Zitat in der Fußzeile in der Form "Vgl. Kramer (2009), S. 2."
\newcommand{\fcite}[2][]{%              
\ifempty{#1}%
\footnote{Vgl.~\citeauthor{#2}~(\citeyear{#2}).}%
\else%
\footnote{Vgl.~\citeauthor{#2}~(\citeyear{#2}),~S.~#1.}%
\fi%
}
%-------------
%% Statt \textcite kann nun auch \citet verwendet werden
\newcommand{\citet}[2][]{%              
    \ifempty{#1}
        \citeauthor{#2}~(\citeyear{#2})
    \else
        \citeauthor{#2}~(\citeyear{#2}),~S.~#1
    \fi
}

%-------------------------------------------
%% Pakete für die richtige Darstellung von Fußnoten
\usepackage[bottom]{footmisc}           % Setzt die Fußnoten ans Ende der Seite
\usepackage{caption}
%\usepackage{savefnmark}
%\footnotesep\baselineskip
%\deffootnote{0.6cm}{1em}{\makebox[0.6cm][l]{\thefootnotemark}}
\setkomafont{footnote}{\fontsize{10pt}{10.5pt}\selectfont}  % Stellt die Fußnotengröße auf 10 ein
\setlength{\footnotesep}{10pt}                              % Sellt den Fußnotenzeilenabstand auf einfachen Zeilenabstand ein

%-------------------------------------------
\usepackage{lipsum}                     % Erstellt blindtext

%-------------------------------------------
%% Deckblatt
\usepackage{pdfpages}                   % Ermöglicht das Einbinden von PDF Dokumenten (wird verwendet für die Einbindung der Aufgabenstellung)
\usepackage[showboxes,absolute]{textpos}   % Ermöglicht das Setzen von Inhalt an einem beliebigen Ort
\usepackage{tikz}

\newcommand\Mygrid{%                        % Erzeugt ein Grid für textpos mit \Mygrid        
\tikz[
  remember picture,
  overlay,
  yscale=-1,
  xstep=\TPHorizModule,ystep=\TPVertModule,
  yshift=0pt,xshift=0pt]
  \draw (current page.north west) grid (current page.south east);}

%  yshift=13pt,xshift=4pt]
%-------------------------------------------
%\usepackage{amsmath}
%\usepackage{adjustbox}

\setlength{\parindent}{0pt}             % Entfernt den Einzug der ersten Zeile

\pagestyle{fancy}                       % Ermöglicht benutzerdefinierte Kopf- und Fußzeilen
\lhead{}
\chead{}
\rhead{\thepage}                        % Seitenangabe oben rechts
\cfoot{} 
\renewcommand{\headrulewidth}{0pt}      % Breite des Dekostriches in der Kopfzeile
\renewcommand{\footrulewidth}{0pt}      % Breite des Dekostriches in der Fußzeile

\newcommand{\titel}{Modellierung und Prognose von Strompreisen durch Neuronale Netze}   % Erstellt eine Variable mit dem Titel der Arbeit
\newcommand{\abgabedatum}{!!13. April 2018!!}   % Erstellt eine Variable mit dem Abgabedatum der Arbeit
\newcommand{\name}{Dennis Albrecht}     % Erstellt eine Variable mit dem Namen des Verfassers

\newcommand\blankpage{%                 % Ermöglicht die Erstellung leerer Seiten mit \blankpage
    \null
    \thispagestyle{empty}%
    \newpage%
}

\renewcommand{\cftsecleader}{\textbf{\cftdotfill{\cftdotsep}}} % Füllt den Leerraum der Sectionenangaben im Inhaltsverzeichnis mit Punkten bis zur Seitenzahl (standardmäßig bei article-class nur ab subsections)

\newcommand{\farbig}[2][red]{\textcolor{#1}{#2}}      % Ermöglicht text mit \farbig{text} rot einzufärben

%-------------------------------------------
%% Einstellungen für die richtige Nummerierung von Fußnoten und Hyperlinks
%% Quelle: https://tex.stackexchange.com/questions/41626/wrapfigure-footnote-hyperlinks
\makeatletter
\newcommand{\wrapfigfoot}{%
\addtocounter{footnote}{+1}%
\addtocounter{Hfootnote}{+1}%
\global\let\Hy@saved@currentHref\@currentHref%
\hyper@makecurrent{Hfootnote}%
\global\let\Hy@footnote@currentHref\@currentHref%
\global\let\@currentHref\Hy@saved@currentHref%
}
\makeatother
%-------------------------------------------