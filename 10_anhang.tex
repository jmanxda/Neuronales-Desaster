% !TEX root = 00_arbeit.tex

%---------------------------------------------------------------------------------
%% Anhang

%-------------------------------------------
%% Resettet den Abbildungszähler
%% Quelle:https://stackoverflow.com/questions/3391540/renumbering-figure-in-latex
\makeatletter 
\renewcommand{\thefigure}{A-\@arabic\c@figure}
\makeatother
\setcounter{figure}{0}

\section{Anhang}

\subsection{XOR bzw. exklusiv Oder}\label{sec:XOR}

\begin{figure}[!h]
\centering
\begin{circuitikz}
\draw (0,0)         node[european xor port] (xor)   {} 
      (xor.in 1)    node[left]                      {A}
      (xor.in 2)    node[left]                      {B}
      (xor.out)     node[right]                     {Y}
      ;
\end{circuitikz}
%\includegraphics{test}
\hspace{1cm}
\begin{tabular}{cc|c}
A & B & Y \\
\hline
0& 0 & 0 \\
0& 1 & 1 \\
1& 0 & 1 \\
1& 1 & 0 
\end{tabular}
\caption{Blockschaltbild des XOR-Gatters links und rechts die zugehörige Wahrheitstabelle.}
\label{fig_a:XOR}
\end{figure}

Das exklusive Oder bzw. XOR-Gatter ist ein Begriff aus dem Bereich der Logik. Mit der Gleichung:%
%
$$Y=\left ( \overline{A}  \land B \right)\lor \left ( A \land \overline{B} \right ).$$
Der Ausgang dieses Gatters ist dann~\glqq 1\grqq~wenn eine ungerade Anzahl an Eingängen auf~\glqq 1\grqq~liegt und die restlichen auf~\glqq 0\grqq. Das Blockschaltbild und die Wahrheitstabelle sind in \autoref{fig_a:XOR} dargestellt.

